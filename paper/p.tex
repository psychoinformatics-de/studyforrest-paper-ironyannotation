\documentclass[10pt,a4paper]{article}
\usepackage{f1000_styles}
\usepackage{units}
\usepackage[colorlinks]{hyperref}
\usepackage{url}
\usepackage[sort,compress]{natbib}
\usepackage{booktabs}

\definecolor{dkgreen}{rgb}{0,0.6,0}
\definecolor{gray}{rgb}{0.5,0.5,0.5}
\definecolor{mauve}{rgb}{0.58,0,0.82}

\begin{document}

\newcommand{\NEventsMin}{46}
\newcommand{\NEventsMax}{56}
\newcommand{\NEventsMedian}{51}
\newcommand{\UniqueSenders}{BOY, BUBBA, DAN, FORREST, FORRESTVO, JENNY, MAN, MRSGUMP, OLDMAN, WOMAN}
\newcommand{\UniqueReceivers}{BUBBA, CROWD, DAN, FORREST, FORREST BUBBA, FORRESTJR, JENNY, MAN, MEN, MRSGUMP, OLDMEN, WOMAN}
\newcommand{\AggTwentyNEvents}{64}
\newcommand{\AggTwentyMeanEventDuration}{6.4}
\newcommand{\AggTwentyMeanEventDistance}{80.6}
\newcommand{\AggTwentyNProofImmediate}{34}
\newcommand{\AggTwentyFKProofImmediate}{0.42}
\newcommand{\AggTwentyNProofPast}{11}
\newcommand{\AggTwentyFKProofPast}{0.62}
\newcommand{\AggTwentyNContradiction}{36}
\newcommand{\AggTwentyFKContradiction}{0.79}
\newcommand{\AggTwentyNLie}{7}
\newcommand{\AggTwentyFKLie}{0.72}
\newcommand{\AggTwentyNIrony}{20}
\newcommand{\AggTwentyFKIrony}{0.90}
\newcommand{\AggTwentyNCueRHETORICALQ}{13}
\newcommand{\AggTwentyFKCueRHETORICALQ}{0.70}
\newcommand{\AggTwentyNCueUNDERSTATEMENT}{2}
\newcommand{\AggTwentyFKCueUNDERSTATEMENT}{0.66}
\newcommand{\AggTwentyNCueEXAGGERATION}{6}
\newcommand{\AggTwentyFKCueEXAGGERATION}{0.59}
\newcommand{\AggTwentyNCueLAUGH}{8}
\newcommand{\AggTwentyFKCueLAUGH}{0.73}
\newcommand{\AggTwentyNCueRAISEDEYEBROW}{13}
\newcommand{\AggTwentyFKCueRAISEDEYEBROW}{0.68}
\newcommand{\AggTwentyNCueEYEROLLING}{2}
\newcommand{\AggTwentyFKCueEYEROLLING}{0.66}
\newcommand{\AggTwentyNCueTEMPOCHANGE}{1}
\newcommand{\AggTwentyFKCueTEMPOCHANGE}{0.48}
\newcommand{\AggTwentyNCueTONECHANGE}{14}
\newcommand{\AggTwentyFKCueTONECHANGE}{0.64}
\newcommand{\AggTwentyNCuePAUSE}{4}
\newcommand{\AggTwentyFKCuePAUSE}{0.51}
\newcommand{\AggSixtyNEvents}{46}
\newcommand{\AggSixtyMeanEventDuration}{5.4}
\newcommand{\AggSixtyMeanEventDistance}{100.5}
\newcommand{\AggSixtyNProofImmediate}{35}
\newcommand{\AggSixtyFKProofImmediate}{0.57}
\newcommand{\AggSixtyNProofPast}{11}
\newcommand{\AggSixtyFKProofPast}{0.71}
\newcommand{\AggSixtyNContradiction}{36}
\newcommand{\AggSixtyFKContradiction}{0.84}
\newcommand{\AggSixtyNLie}{7}
\newcommand{\AggSixtyFKLie}{0.76}
\newcommand{\AggSixtyNIrony}{20}
\newcommand{\AggSixtyFKIrony}{0.91}
\newcommand{\AggSixtyNCueRHETORICALQ}{13}
\newcommand{\AggSixtyFKCueRHETORICALQ}{0.89}
\newcommand{\AggSixtyNCueUNDERSTATEMENT}{2}
\newcommand{\AggSixtyFKCueUNDERSTATEMENT}{0.79}
\newcommand{\AggSixtyNCueEXAGGERATION}{6}
\newcommand{\AggSixtyFKCueEXAGGERATION}{0.65}
\newcommand{\AggSixtyNCueLAUGH}{8}
\newcommand{\AggSixtyFKCueLAUGH}{0.89}
\newcommand{\AggSixtyNCueRAISEDEYEBROW}{14}
\newcommand{\AggSixtyFKCueRAISEDEYEBROW}{0.73}
\newcommand{\AggSixtyNCueEYEROLLING}{2}
\newcommand{\AggSixtyFKCueEYEROLLING}{0.79}
\newcommand{\AggSixtyNCueTEMPOCHANGE}{1}
\newcommand{\AggSixtyFKCueTEMPOCHANGE}{0.58}
\newcommand{\AggSixtyNCueTONECHANGE}{14}
\newcommand{\AggSixtyFKCueTONECHANGE}{0.67}
\newcommand{\AggSixtyNCuePAUSE}{4}
\newcommand{\AggSixtyFKCuePAUSE}{0.58}
\newcommand{\AggHundredNEvents}{36}
\newcommand{\AggHundredMeanEventDuration}{4.6}
\newcommand{\AggHundredMeanEventDistance}{131.6}
\newcommand{\AggHundredNProofImmediate}{28}
\newcommand{\AggHundredFKProofImmediate}{0.79}
\newcommand{\AggHundredNProofPast}{10}
\newcommand{\AggHundredFKProofPast}{0.79}
\newcommand{\AggHundredNContradiction}{32}
\newcommand{\AggHundredFKContradiction}{1.00}
\newcommand{\AggHundredNLie}{6}
\newcommand{\AggHundredFKLie}{0.85}
\newcommand{\AggHundredNIrony}{19}
\newcommand{\AggHundredFKIrony}{0.93}
\newcommand{\AggHundredNCueRHETORICALQ}{11}
\newcommand{\AggHundredFKCueRHETORICALQ}{0.96}
\newcommand{\AggHundredNCueUNDERSTATEMENT}{2}
\newcommand{\AggHundredFKCueUNDERSTATEMENT}{0.79}
\newcommand{\AggHundredNCueEXAGGERATION}{6}
\newcommand{\AggHundredFKCueEXAGGERATION}{0.68}
\newcommand{\AggHundredNCueLAUGH}{7}
\newcommand{\AggHundredFKCueLAUGH}{0.94}
\newcommand{\AggHundredNCueRAISEDEYEBROW}{11}
\newcommand{\AggHundredFKCueRAISEDEYEBROW}{0.81}
\newcommand{\AggHundredNCueEYEROLLING}{1}
\newcommand{\AggHundredFKCueEYEROLLING}{1.00}
\newcommand{\AggHundredNCueTEMPOCHANGE}{1}
\newcommand{\AggHundredFKCueTEMPOCHANGE}{0.74}
\newcommand{\AggHundredNCueTONECHANGE}{12}
\newcommand{\AggHundredFKCueTONECHANGE}{0.76}
\newcommand{\AggHundredNCuePAUSE}{4}
\newcommand{\AggHundredFKCuePAUSE}{0.56}


\title{Lies, irony, and contradiction — an annotation of semantic conflict in the movie ``Forrest Gump''}

\author[1,2]{Michael~Hanke}
\author[1]{Pierre~Ibe}

\affil[1]{Psychoinformatics lab, Department of Psychology, University of Magdeburg, Universit\"{a}tsplatz 2, 39106 Magdeburg, Germany}
\affil[2]{Center for Behavioral Brain Sciences, Magdeburg, Germany}
\maketitle
\thispagestyle{fancy}

\begin{abstract}
% Abstracts should be up to 300 words and provide a succinct summary of the
% article. Although the abstract should explain why the article might be
% interesting, care should be taken not to inappropriately over-emphasise the
% importance of the work described in the article. Citations should not be used
% in the abstract, and the use of abbreviations should be minimized.

  Here we extend the information on the structure of the core stimulus of the
  studyforrest project (http://studyforrest.org) with a description of semantic
  conflict in the ``Forrest Gump'' movie.  Three observers annotated the movie
  independently regarding episodes with portrayal of lies, irony or sarcasm.
  We present frequency statistics, and inter-observer reliability measures that
  qualify and quantify semantic conflict in the stimulus.  While the number of
  identified events is limited, this annotation nevertheless enriches the
  knowledge about the complex high-level structure of this stimulus, and can
  help to evalute its utility for future studies, and the usability of the
  existing brain imaging data regarding this aspect of cognition.

\end{abstract}

\section*{Introduction}
%The format of the main body of the article is flexible: it should be concise
%and in the format most appropriate to displaying the content of the article.

%A brief summary of how this work was motivated and how it links to existing
%and future work.

Detection of semantic conflict is an important cognitive skill for human social
interaction. It is required to identify lies (false statements made with the
intention to deceive) but also to correctly interpret stylistic devices — such
as sarcasm and irony (statements with direct meaning that is the opposite
\citep{AEH+03} or contrary \citep{Han2004} to the implied semantic content). As
the interpretation of such events is highly context dependent, it is difficult
to study how the brain processes these in the context of real-life like
interactions in complex natural environments.

In this study, we explored occurrences of semantic conflict in the core stimulus
of the studyforrest project (\url{http://studyforrest.org}) — the motion picture
``Forrest Gump'' — in order to evaluate whether the available brain imaging data
\citep{HBI+14,HAK+16} can be readily used to study this aspect of cognition.
We annotated the presence of contradictory statements, including lies and
ironic statements, as well as the portrayal of cues, such as exaggeration or
raised eyebrows, that are often associated with making ironic statements.
Additionally, we recorded the context that allowed observers to classify an
event as contradictory.

Depending on the exact criterion used for identifying events across observers,
we found only between \AggTwentyNEvents\ and \AggHundredNEvents\ occurrences of
semantic conflict or portrayal of irony cues in the entire movie stimulus.
These are likely insufficient numbers for an investigation based on these data
alone. However, these new annotations nevertheless contribute to a more
comprehensive description of this complex movie stimulus
\citep{LRS+2015,HH2016} and may be useful as confound variables in subsequent
studies.


\section*{Materials and methods}

\subsection*{Stimulus}

The annotated stimulus was a slightly shortened ($\approx$\unit[2]{h}) version
of the movie Forrest~Gump (R.~Zemeckis, Paramount Pictures, 1994), with a dubbed
German soundtrack, and is identical to the audio-visual movie annotated in
\cite{LRS+2015,HH2016}. Further details on this particular movie cut, and how
to reproduce it from commercially available sources, are available in
\cite{HAK+16}.

\subsection*{Observers}

Three observers (all female, age 19–20) independently annotated the movie. They
were also involved in the development of the concept for this annotation.

\subsection*{Procedure}

Observers were instructed to watch the movie from beginning to end, replaying
scenes as often as required, and to detect two types of events: 1) whenever a
verbal statement is made that contradicts with either the immediate context or
with the viewer's body-of-knowledge at this point in the movie, or 2) whenever
one or more cues associated with irony (predefined list, see below) are
portrayed. In either case, observers had to describe the event by specifying its
properties via a number of variable settings in a spreadsheet. The software
video player VLC (\url{http://www.videolan.org/vlc}) was used to watch and
navigate through the movie.


\section*{Data legend}

For each annotated event, a total of 10 properties were recorded, each of which
are described in the following sections.

\paragraph{Start and end} The duration of each event is recorded in
\texttt{start} and \texttt{end} as the number of seconds from movie start (no
sub-second precision, due to limitations of the video player time display). The
time-points correspond to the onset and offset of the respective evidence. Both
times can be identical in the case of events with less than one second duration.
For contradictory statements, the duration covers the time from the onset of
evidence of a contradiction until the end of the statement.

\paragraph{Sender and receiver} The identity of a character making a
contradictory statement or portraying an irony cue is encoded in
\texttt{sender} using character labels listed in \citep{LRS+2015}.
In the case that the respective statement is directed to another present movie
character, its identity is encoded in \texttt{receiver}.

\paragraph{Evidence of a contradiction}

The \texttt{contradiction} flag indicates the presence of a contradiction
in an event (1:~present, 0:~absent). The variable \texttt{proof} qualifies
if the current or previous events provide the viewer with information to allow
the detection of this contradiction (see Table
\ref{tab:annotationcodes}). If \texttt{proof} is empty, the movie itself does
not contain such information (e.g. a common sense contradiction).

\paragraph{Irony cues} The variable \texttt{cues} contains a space-separated
list of labels for all irony cues present in a particular events. See Table
\ref{tab:annotationcodes} for a description of all possible labels.

\paragraph{Event category} The \texttt{category} variable classifies events
into lies, ironic statements, and other events (value empty).

\paragraph{Intention} Two more variables encode whether a contradiction was used
deliberately and whether this was noticed by the receiver. The variable
\texttt{intended} encodes the presence of evidence for deliberate use (1:~yes,
0:~no). The variable is empty if there is no evidence for either case. The
second variable \texttt{intention\_decoded} encodes, in the same way, whether a
potential receiver noticed a deliberate ironic statement or lie.

\begin{table*}
  \centering
  \begin{tabular}{lp{10cm}}
    \toprule\\
    \textbf{Code} & \textbf{Description} \\
    \\\midrule\\
    \textit{Cues}\\
    RHETORICALQ & a statement that is or contains a rhetorical question \\
    UNDERSTATEMENT & a fact is stated in a weakened manner \\
    EXAGGERATION & a fact is overstated \\
    LAUGH & a statement accompanied by laughter \\
    RAISEDEYEBROW & character makes a statement while raising an eyebrow \\
    EYEROLLING & character makes a statement while rolling his/her eyes \\
    TEMPOCHANGE & a change of the speech tempo within a sentence or compared to previous sentences\\
    TONECHANGE & a change of the tone, pitch, or volume within a sentence or compared to previous sentences \\
    % AKA, the Christopher Walken effect ;-)
    PAUSE & an intentionally placed (yet still unexpected) or unnatural pause within a sentence \\
    \\\midrule\\
    \textit{Contradiction proof}\\
    IMMEDIATE & a statement that contradicts with information in the immediate context \\
    PAST &  a statement that contradicts with previous actions or information given in the past\\
    \\\midrule\\
    \textit{Conflict category}\\
    IRONY & an ironic statement (a statement with a direct meaning that is the opposite
            or contrary to its implied semantic content)\\
    LIE & a lie (a false statement made with the intention to deceive)\\
    \\\bottomrule
  \end{tabular}
  \caption{Definitions for all annotation codes.}
  \label{tab:annotationcodes}
\end{table*}

\subsection*{Dataset content}

The released annotation are three, text-based, comma-separated-value (CSV)
formatted tables (\texttt{data/o??.csv}), one for each observer.

The source code for all descriptive statistics included in this paper is
available in \texttt{code/descriptive\_stats.py} (Python script).

\paragraph{Dataset 1.}
CSV tables with of events of semantic conflict, and occurences of irony cues
in the motion picture “Forrest Gump”.

\paragraph{Dataset 2.}
Python script to compute all descriptive statistics presented in the
Data Note manuscript from the released annotations.

\section*{Dataset validation}
%Use section and subsection commands to organize your document. \LaTeX{}
%handles all the formatting and numbering automatically. Use ref and label
%commands for cross-references.

%This section is not essential for Web Tool papers.
%For Data Articles, no analysis of the data, results or conclusions should be
%included and so this section should not be completed.


We used an automated procedure to check the annotation records of individual
observers for errors or potential problems. Observers submitted their
annotations in tabular form to a script that generated a list of error and
warning messages. Using this feedback, observers double-checked their
annotations as often as necessary until no objective errors were found and all
warning messages were confirmed to be false positives. The tests included, for
example, plausibility of timing information (no end time before the respective
start time) or the presence of unknown condition labels.

In order to assess inter-observer agreement of annotations, we used a two-step
approach. First, the temporal location of events depicting \textit{any}
relevant property were determined by comparing annotation timing across
observers. The columns in Table \ref{tab:validation} report agreement
statistics for events defined by at least one, two, or all three observers
recording an annotation for the same \texttt{sender} at the same time. In the
case that individual observers reported events of different length, or with
only partially overlapping duration, only the time-windows with the minimum
number of observers reporting an event were considered.

In the second step, we computed Fleiss' Kappa \citep{Fle71} for each individual
property of an annotation separately with respect to being consistently
assigned or non-assigned to the identified events (Table \ref{tab:validation}).
We observe increasing inter-observer agreement of all annotated properties with
increasing agreement of annotation timing, approaching ``substantial'' or
``almost perfect'' agreement — according to the conventions put forth by
\cite{LK77}.

\begin{table*}
  \centering
  \begin{tabular}{lrrr}
    \toprule\\
    & \multicolumn{3}{c}{\textbf{Event location min. agreement}} \\
    & \textbf{33\%} & \textbf{66\%} & \textbf{100\%} \\
    \\\midrule\\
    Number of events & \AggTwentyNEvents & \AggSixtyNEvents & \AggHundredNEvents \\
    Mean event duration (s) & \AggTwentyMeanEventDuration & \AggSixtyMeanEventDuration & \AggHundredMeanEventDuration \\
    Mean event distance (s) & \AggTwentyMeanEventDistance & \AggSixtyMeanEventDistance & \AggHundredMeanEventDistance \\
    \\\midrule\\
    \multicolumn{4}{l}{\textit{Event types}}\\
    \multicolumn{4}{l}{\small[majority vote counts and inter-observer agreement (Fleiss' Kappa)]}\\\\
    Contradiction & \AggTwentyNContradiction\ (\AggTwentyFKContradiction) & \AggSixtyNContradiction\ (\AggSixtyFKContradiction) & \AggHundredNContradiction\ (\AggHundredFKContradiction)\\
    Irony & \AggTwentyNIrony\ (\AggTwentyFKIrony) & \AggSixtyNIrony\ (\AggSixtyFKIrony) & \AggHundredNIrony\ (\AggHundredFKIrony)\\
    Lie & \AggTwentyNLie\ (\AggTwentyFKLie) & \AggSixtyNLie\ (\AggSixtyFKLie) & \AggHundredNLie\ (\AggHundredFKLie)\\
    \\\midrule\\
    \multicolumn{4}{l}{\textit{Contradiction evidence}}\\
    \multicolumn{4}{l}{\small[majority vote counts and inter-observer agreement (Fleiss' Kappa)]}\\\\
    Immediate context & \AggTwentyNProofImmediate\ (\AggTwentyFKProofImmediate) & \AggSixtyNProofImmediate\ (\AggSixtyFKProofImmediate) & \AggHundredNProofImmediate\ (\AggHundredFKProofImmediate)\\
    Memory & \AggTwentyNProofPast\ (\AggTwentyFKProofPast) & \AggSixtyNProofPast\ (\AggSixtyFKProofPast) & \AggHundredNProofPast\ (\AggHundredFKProofPast)\\
    \\\midrule\\
    \multicolumn{4}{l}{\textit{Cue type occurrences}}\\
    \multicolumn{4}{l}{\small[majority vote counts and inter-observer agreement (Fleiss' Kappa)]}\\\\
    Rhetorical question & \AggTwentyNCueRHETORICALQ\ (\AggTwentyFKCueRHETORICALQ) & \AggSixtyNCueRHETORICALQ\ (\AggSixtyFKCueRHETORICALQ) & \AggHundredNCueRHETORICALQ\ (\AggHundredFKCueRHETORICALQ)\\
    Understatement
    & \AggTwentyNCueUNDERSTATEMENT\ (\AggTwentyFKCueUNDERSTATEMENT) & \AggSixtyNCueUNDERSTATEMENT\ (\AggSixtyFKCueUNDERSTATEMENT) & \AggHundredNCueUNDERSTATEMENT\ (\AggHundredFKCueUNDERSTATEMENT)\\
    Exaggeration
    & \AggTwentyNCueEXAGGERATION\ (\AggTwentyFKCueEXAGGERATION) & \AggSixtyNCueEXAGGERATION\ (\AggSixtyFKCueEXAGGERATION) & \AggHundredNCueEXAGGERATION\ (\AggHundredFKCueEXAGGERATION)\\
    Laughter
    & \AggTwentyNCueLAUGH\ (\AggTwentyFKCueLAUGH) & \AggSixtyNCueLAUGH\ (\AggSixtyFKCueLAUGH) & \AggHundredNCueLAUGH\ (\AggHundredFKCueLAUGH)\\
    Raised eyebrow
    & \AggTwentyNCueRAISEDEYEBROW\ (\AggTwentyFKCueRAISEDEYEBROW) & \AggSixtyNCueRAISEDEYEBROW\ (\AggSixtyFKCueRAISEDEYEBROW) & \AggHundredNCueRAISEDEYEBROW\ (\AggHundredFKCueRAISEDEYEBROW)\\
    Eye rolling
    & \AggTwentyNCueEYEROLLING\ (\AggTwentyFKCueEYEROLLING) & \AggSixtyNCueEYEROLLING\ (\AggSixtyFKCueEYEROLLING) & \AggHundredNCueEYEROLLING\ (\AggHundredFKCueEYEROLLING)\\
    Tempo change
    & \AggTwentyNCueTEMPOCHANGE\ (\AggTwentyFKCueTEMPOCHANGE) & \AggSixtyNCueTEMPOCHANGE\ (\AggSixtyFKCueTEMPOCHANGE) & \AggHundredNCueTEMPOCHANGE\ (\AggHundredFKCueTEMPOCHANGE)\\
    Tone change
    & \AggTwentyNCueTONECHANGE\ (\AggTwentyFKCueTONECHANGE) & \AggSixtyNCueTONECHANGE\ (\AggSixtyFKCueTONECHANGE) & \AggHundredNCueTONECHANGE\ (\AggHundredFKCueTONECHANGE)\\
    Pause
    & \AggTwentyNCuePAUSE\ (\AggTwentyFKCuePAUSE) & \AggSixtyNCuePAUSE\ (\AggSixtyFKCuePAUSE) & \AggHundredNCuePAUSE\ (\AggHundredFKCuePAUSE)\\
    \\\bottomrule
  \end{tabular}

  \caption{
    %
    Annotation inter-observer agreement statistics. Number of events and
    categorization agreement are presented for three levels
    of inter-observer agreement on the temporal location and the performing
    movie character. The number of events for any particular event property
    are determined by majority vote across observers, i.e. an event is counted
    when more observers indicate the presence of a property than its absence.
    Exhaustive technical detail on the statistical analysis can be found in the
    \texttt{descriptive\_stats.py} Python script.
    %
  }

  \label{tab:validation}
\end{table*}



\section*{Data and software availability}

\texttt{This section will be auto-generated.}

In addition, released data, code, and manuscript sources are also available on
Github
(\url{https://github.com/psychoinformatics-de/studyforrest-paper-ironyannotation}).


\section*{Author contributions}
%In order to give appropriate credit to each author of an article, the
%individual contributions of each author to the manuscript should be detailed
%in this section. We recommend using author initials and then stating briefly
%how they contributed.

MH contributed to the design of the annotation effort, performed the dataset
validation, and wrote the paper; PI contributed to the design, coordinated the
annotation effort, and wrote the paper.

\section*{Competing Interests}
No competing interests were disclosed.

\section*{Grant Information}

Michael Hanke was supported by funds from the German federal state of
Saxony-Anhalt and the European Regional Development Fund (ERDF), Project:
Center for Behavioral Brain Sciences.

\section*{Acknowledgements}
%This section should acknowledge anyone who contributed to the research or the
%article but who does not qualify as an author based on the criteria provided
%earlier (e.g. someone or an organisation that provided writing assistance).
%Please state how they contributed; authors should obtain permission to
%acknowledge from all those mentioned in the Acknowledgements section.  Please
%do not list grant funding in this section (this should be included in the
%Grant information section - See above).

We are grateful to Denise Naumann, Marisela Markarian, Jasmin Billowie, and
Susann Bergmann for their contributions to the design and the execution of the
annotation effort. We also appreciate Alex Waite for his seemingly unending
willingness to edit papers.


%\nocite{*}
{\small\bibliographystyle{unsrt}
\bibliography{references}}

\end{document}
